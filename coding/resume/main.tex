%% start of file `template-zh.tex'.
%% Copyright 2006-2013 Xavier Danaux (xdanaux@gmail.com).
%
% This work may be distributed and/or modified under the
% conditions of the LaTeX Project Public License version 1.3c,
% available at http://www.latex-project.org/lppl/.


\documentclass[11pt,a4paper,sans]{moderncv}   % possible options include font size ('10pt', '11pt' and '12pt'), paper size ('a4paper', 'letterpaper', 'a5paper', 'legalpaper', 'executivepaper' and 'landscape') and font family ('sans' and 'roman')

% moderncv 主题
\moderncvstyle{classic}
%\moderncvstyle{banking}                        % 选项参数是 ‘casual’, ‘classic’, ‘oldstyle’ 和 ’banking’
%\moderncvcolor{blue}                          % 选项参数是 ‘blue’ (默认)、‘orange’、‘green’、‘red’、‘purple’ 和 ‘grey’
\moderncvcolor{black}                          % 选项参数是 ‘blue’ (默认)、‘orange’、‘green’、‘red’、‘purple’ 和 ‘grey’
%\nopagenumbers{}                             % 消除注释以取消自动页码生成功能

% 字符编码
\usepackage[utf8]{inputenc}                   % 替换你正在使用的编码
\usepackage{CJKutf8}

% 更改typewriter字体
\usepackage{DejaVuSansMono}
%\renewcommand*{\ttdefault}{dejavumono}
%\usepackage[scaled=0.85]{beramono}
%\usepackage{times}

% 调整页面出血
\usepackage[top=20mm,bottom=28mm,left=24mm,right=24mm]{geometry}
%\setlength{\hintscolumnwidth}{3cm}           % 如果你希望改变日期栏的宽度

% 个人信息
%\name{韬}{林}
\name{林}{韬}
\title{简历}                     % 可选项、如不需要可删除本行
%\address{石景山区玉泉路19号乙高能物理研究所主楼B406}{100049 北京}            % 可选项、如不需要可删除本行
\address{石景山区玉泉路19号乙多学科大楼}{100049 北京}            % 可选项、如不需要可删除本行
\phone[mobile]{137~1641~2993}              % 可选项、如不需要可删除本行
\phone[fixed]{010-8823~4010}               % 可选项、如不需要可删除本行
%\phone[fax]{+3~(456)~789~012}                 % 可选项、如不需要可删除本行
\email{lintao51@gmail.com}                    % 可选项、如不需要可删除本行
\homepage{https://github.com/mirguest/}                  % 可选项、如不需要可删除本行
\extrainfo{https://bitbucket.org/lintao/}                 % 可选项、如不需要可删除本行
\photo[96pt]{qr-github-com}                  % ‘64pt’是图片必须压缩至的高度、‘0.4pt‘是图片边框的宽度 (如不需要可调节至0pt)、’picture‘ 是图片文件的名字;可选项、如不需要可删除本行

% 显示索引号;仅用于在简历中使用了引言
%\makeatletter
%\renewcommand*{\bibliographyitemlabel}{\@biblabel{\arabic{enumiv}}}
%\makeatother

% 分类索引
%\usepackage{multibib}
%\newcites{book,misc}{{Books},{Others}}
%----------------------------------------------------------------------------------
%            内容
%----------------------------------------------------------------------------------
\begin{document}
\begin{CJK}{UTF8}{gbsn}                       % 详情参阅CJK文件包
\maketitle

\section{教育背景}
\cventry{2011年 -- 2016年~~{}}{博士}{中国科学院大学}{北京}{粒子物理实验 -- 高能物理计算方向}{曾获三好学生, 研究生国家奖学金}  % 第3到第6编码可留白
\cventry{2007年 -- 2011年~~{}}{学士}{兰州大学}{兰州}{物理学基地班}{保送研究生, 曾获校一等奖学金}

%\section{毕业论文}
%\cvitem{题目}{\emph{题目}}
%\cvitem{导师}{导师}
%\cvitem{说明}{\small 论文简介}

\section{工作背景}
% \subsection{助理研究员}
\cventry{2021年 -- 至今}{副研究员}{中科院高能物理研究所}{北京}{}{
}


% \subsection{助理研究员}
\cventry{2018年 -- 2021年}{助理研究员}{中科院高能物理研究所}{北京}{}{
}

% \subsection{博士后}
\cventry{2016年 -- 2018年}{博士后}{中科院高能物理研究所}{北京}{}{
}

% \subsection{硕博生}
\cventry{2012年 -- 2016年}{硕博连读}{中科院高能物理研究所}{北京}{}{
% \begin{itemize}
% \item 参与国内大型实验(BESIII, Daya Bay, JUNO)中离线软件环境的维护(基于Bash和Python)
%     \begin{itemize}%
%     \item 参与BESIII实验的离线软件环境的升级与维护; 
%     \item 负责Daya Bay实验中Trac+SVN的升级与维护;
%     \item 负责JUNO实验中mail list, Trac+SVN的部署及维护; 同时维护合作组数据库等信息;
%     \item 开发基于bash的junoenv, 用于部署JUNO离线软件及外部库
%     \end{itemize}
% \item 离线数据处理软件平台和探测器模拟软件的开发 (基于C++和Python)
%     \begin{itemize}%
%     \item 参与BESIII网格系统的开发, 负责开发基于DIRAC的数据传输系统;
%     \item 参与JUNO离线软件中核心框架SNiPER的开发以及框架的python binding;
%     \item 负责开发JUNO探测器模拟软件. 该软件基于Geant4, 并与离线框架整合;
%     \item 研究快速模拟技术. 针对探测器模拟中宇宙线事例运行时间长, 占用内存大, 进行了优化.
%           完成基于Voxel Method的快速模拟方法, 并研究了基于CUDA技术的方法.
%     \end{itemize}
% \end{itemize}
}
% \subsection{本科生}
\cventry{2011年2月 -- 6月~~}{本科}{中科院高能物理研究所}{北京}{}{
% \begin{itemize}
% \item 参与Daya Bay软件开发, 完成基于Python的作业提交及Bookkeeping系统
% \end{itemize}
}




\clearpage\end{CJK}
\end{document}


%% 文件结尾 `template-zh.tex'.
